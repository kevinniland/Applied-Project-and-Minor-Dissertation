\chapter{Conclusion}
This chapter serves as a conclusion to the project. In it, the objectives of the project are analysed again briefly to evaluate whether or not they were met. It also analyses any improvements and/or changes that could be added/made to the project if it was to be developed again.

\section{Objectives}
One of the objectives of this project was to make an application that had the capabilities to visualize several different sorting algorithms, from start to finish. This application is intended and can be useful in an educational context, where users can use it to see how the programmed algorithms perform a sort of a provided array of elements. This was achieved by highlighting the elements currently chosen to be sorted and repeating this until the entire array is sorted. 
\par
\bigskip
Another one of the objectives of the project was to provide users of a system with the ability to register for an account, log in, and have the ability to access a screen record API and have the ability to upload the recording of the sort to a database, which can then be viewed by all users. The user registration/log-in system was achieved through a combination of Python and MongoDB. Creating a screen recording was done through the use of an external API, Screen Flow. Firebase was subsequently used to store these screen recordings.

\section{Reflection}
Overall, the objectives of the project were met and done to a standard that I was happy with. However, as with all software projects, there would be things I would do differently, incorporate different technologies to make a certain feature more robust, etc.

\subsection{Downfalls and Improvements}
If the project was to be repeated, a number of changes and improvements could be made. While user authentication does play a role, this could be further enhanced in several different ways such as further enhancing the viewing of all sorts aspect of the project. The user could have the option of uploading a dataset in addition to uploading a saved sort. The uploading of sorts could have been done in a better way. For instance, the Screen Record API used could have been integrated with the application more. This was attempted but due to the way the API was programmed, it does not seem like integrating it with this application in a more cohesive manner was possible at the time. 
\bigskip

Another improvement could have been to include more sorting algorithms. The application could also have shown a visualization of the code itself. For example, when a certain sorting algorithm is picked, a window would appear with the appropriate code for the selected sorting algorithm. When the array of elements is being sorted, the appropriate logic in the code could be highlighted (such as the logic for swapping elements, comparison of different elements, etc.).
\bigskip

A downfall of the application could be the user is unable to randomly generate an array of a certain size. The user is also unable to control the speed of the sort itself.

\subsection{Additions}
As this is application that visualizes algorithms, a viable addition to the application could be to incorporate pathfinding algorithms. Pathfinding algorithms were recently covered on the course in the Artificial Intelligence module. This would come with it's own challenges, however...

\subsection{Overall}
Overall, this was a enjoyable project to work on and provided great experience in many different areas, such as undertaking and developing a project over a prolonged period of time, working with new technologies, and creating an application that can be deployed and used by multiple users.